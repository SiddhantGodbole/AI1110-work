\let\negmedspace\undefined
\let\negthickspace\undefined
\documentclass[journal,12pt,onecolumn]{IEEEtran}
\usepackage{multicol}

\usepackage{cite}
\usepackage{amsmath,amssymb,amsfonts,amsthm}
\usepackage{algorithmic}
\usepackage{graphicx}
\usepackage{textcomp}
\usepackage{xcolor}
\usepackage{txfonts}
\usepackage{listings}
\usepackage{enumitem}
\usepackage{mathtools}
\usepackage{gensymb}
\usepackage[breaklinks=true]{hyperref}
\usepackage{tkz-euclide} % loads  TikZ and tkz-base
\usepackage{listings}
\usepackage{amsmath}
\newcommand*\textfrac[2]{
  \frac{\text{#1}}{\text{#2}}
}
\DeclareMathOperator*{\Res}{Res}
\renewcommand\thesection{\arabic{section}}
\renewcommand\thesubsection{\thesection.\arabic{subsection}}
\renewcommand\thesubsubsection{\thesubsection.\arabic{subsubsection}}

\renewcommand\thesectiondis{\arabic{section}}
\renewcommand\thesubsectiondis{\thesectiondis.\arabic{subsection}}
\renewcommand\thesubsubsectiondis{\thesubsectiondis.\arabic{subsubsection}}

% correct bad hyphenation here
%\hyphenation{optical-networks-semiconductor}
%\def\inputGnumericTable{}                                

\lstset{
%language=C,
frame=single, 
breaklines=true,
columns=fullflexible
}


\newtheorem{theorem}{Theorem}[section]
\newtheorem{problem}{Problem}
\newtheorem{proposition}{Proposition}[section]
\newtheorem{lemma}{Lemma}[section]
\newtheorem{corollary}[theorem]{Corollary}
\newtheorem{example}{Example}[section]
\newtheorem{definition}[problem]{Definition}

\newcommand{\BEQA}{\begin{eqnarray}}
\newcommand{\EEQA}{\end{eqnarray}}
\newcommand{\define}{\stackrel{\triangle}{=}}
\newcommand\tab[1][1cm]{\hspace*{#1}}
\bibliographystyle{IEEEtran}
\providecommand{\mbf}{\mathbf}
\providecommand{\pr}[1]{\ensuremath{\Pr\left(#1\right)}}
\providecommand{\qfunc}[1]{\ensuremath{Q\left(#1\right)}}
\providecommand{\sbrak}[1]{\ensuremath{{}\left[#1\right]}}
\providecommand{\lsbrak}[1]{\ensuremath{{}\left[#1\right.}}
\providecommand{\rsbrak}[1]{\ensuremath{{}\left.#1\right]}}
\providecommand{\brak}[1]{\ensuremath{\left(#1\right)}}
\providecommand{\lbrak}[1]{\ensuremath{\left(#1\right.}}
\providecommand{\rbrak}[1]{\ensuremath{\left.#1\right)}}
\providecommand{\cbrak}[1]{\ensuremath{\left\{#1\right\}}}
\providecommand{\lcbrak}[1]{\ensuremath{\left\{#1\right.}}
\providecommand{\rcbrak}[1]{\ensuremath{\left.#1\right\}}}
\theoremstyle{remark}
\newtheorem{rem}{Remark}
\newcommand{\sgn}{\mathop{\mathrm{sgn}}}
\providecommand{\abs}[1]{\left\vert#1\right\vert}
\providecommand{\res}[1]{\Res\displaylimits_{#1}} 
\providecommand{\norm}[1]{\left\lVert#1\right\rVert}
\providecommand{\mtx}[1]{\mathbf{#1}}
\providecommand{\mean}[1]{E\left[ #1 \right]}
\providecommand{\fourier}{\overset{\mathcal{F}}{ \rightleftharpoons}}
\providecommand{\system}{\overset{\mathcal{H}}{ \longleftrightarrow}}
\newcommand{\solution}{\noindent \textbf{Solution: }}
\newcommand{\cosec}{\,\text{cosec}\,}
\providecommand{\dec}[2]{\ensuremath{\overset{#1}{\underset{#2}{\gtrless}}}}
\newcommand{\myvec}[1]{\ensuremath{\begin{pmatrix}#1\end{pmatrix}}}
\newcommand{\mydet}[1]{\ensuremath{\begin{vmatrix}#1\end{vmatrix}}}

\begin{document}
%
\vspace{3cm}
%
\title{
Assignment 1

\large{
AI1110 : Probability And Random Variables}
\author{Siddhant Godbole\\CS22BTECH11054}} 
%       
%
% make the title area
%
%
%
%
%
%        
\maketitle\bigskip
%\newpage
%\tableofcontents
%\renewcommand{\thefigure}{\theenumi}
%\renewcommand{\thetable}{\theenumi}
%\renewcommand{\theequation}{\theenumi}
%
%
%
\begin{multicols}{2}
\text{ }
\\
\textbf{lemh-}
\textbf{12.13.1.8 -}
\\
\smallskip
\textbf{Question:}
A die is thrown three times,%
\medskip
\newline
%
E : 4 appears on the third toss,%
\newline
F : 6 and 5 appears respectively \newline
\tab on first two tosses%
\bigskip
%
\text{ }
\newline
\textbf{Solution:}
A fair dice is tossed thrice.\medskip
\newline
There are three differnt ordered outcome each with values from 1 to 6 with equal probability.
\medskip
\newline
Let $X$ be a random variable which takes the values 1 , 2 , 3 , 4 , 5 and 6.
\medskip
\newline
$P_1 , P_2$ and $P_3$are probablities \newline connected to respective three dice rolls.
\medskip
\newline
A fair die gives equal (1/6) probability for any X .
\bigskip
\newline
S being the set of the sample space.
\begin{align}
	 S = \{ 1, 2, 3, 4, 5, 6 \}
\end{align}\columnbreak
\begin{align}
	P(E) =   P_{1}  ( S) \cdot P_2 ( S) \cdot P_3( X = 4 ) 
\end{align}
\begin{align}
\therefore 	P(E) = 1\cdot 1\cdot  (1/6)     	
\end{align}
\begin{align}
	P(F) = P_1(X = 6)\cdot P_2( X = 5) \cdot P_3(S)
\end{align}
\begin{align}
\therefore     	P(F) = (1/6) \cdot (1/6) \cdot 1 \\= (1/36)
\end{align}
%
\begin{center}
\_\_\_\_dice-dependence
\newline
\begin{tabular}
{|c | c c c | c |}\hline
Base & &Contribution && Final\\\hline\hline
X & D1 & D2 & D3 & P \\[0.5ex]\hline
E & 1 & 1 & 1/6 & 1/6 \\[0.5ex]\hline
F & 1/6 & 1/6 & 1 & 1/36\\[0.5ex]\hline
\end{tabular}
\end{center}
%
So, probability of 
\newline
\textbf{E} : 4 appears on the third toss  is : 1/6 or 0.167 or 16.7\%
\medskip
\newline
\textbf{F} : 6 and 5 appears respectively on first two tosses is : 1/36 or 0.0278 or 2.78\%  
\end{multicols}
\end{document}  
